% Verbal description for system INTFX (INTFX_desc.tex)
% Generated by MTT on Fri Aug 15 09:53:16 BST 1997.

% %%%%%%%%%%%%%%%%%%%%%%%%%%%%%%%%%%%%%%%%%%%%%%%%%%%%%%%%%%%%%%%
% %% Version control history
% %%%%%%%%%%%%%%%%%%%%%%%%%%%%%%%%%%%%%%%%%%%%%%%%%%%%%%%%%%%%%%%
% %% $Id: INTFX_desc.tex,v 1.1 2000/12/28 17:25:34 peterg Exp $
% %% $Log: INTFX_desc.tex,v $
% %% Revision 1.1  2000/12/28 17:25:34  peterg
% %% To RCS
% %%
% %% Revision 1.1  1997/08/24 11:20:18  peterg
% %% Initial revision
% %%
% %%%%%%%%%%%%%%%%%%%%%%%%%%%%%%%%%%%%%%%%%%%%%%%%%%%%%%%%%%%%%%%

   The acausal bond graph of system \textbf{INTFX} is
   displayed in Figure \Ref{INTFX_abg} and its label
   file is listed in Section \Ref{sec:INTFX_lbl}.
   The subsystems are listed in Section \Ref{sec:INTFX_sub}.

\textbf{INTFX} is a two-port component where the effort on port [out]
   is the integral of the flow on port [in].
