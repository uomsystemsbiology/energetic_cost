% Verbal description for system TwoLinkPX (TwoLinkPX_desc.tex)
% Generated by MTT on Fri Jun 13 16:30:23 BST 1997.

% %%%%%%%%%%%%%%%%%%%%%%%%%%%%%%%%%%%%%%%%%%%%%%%%%%%%%%%%%%%%%%%
% %% Version control history
% %%%%%%%%%%%%%%%%%%%%%%%%%%%%%%%%%%%%%%%%%%%%%%%%%%%%%%%%%%%%%%%
% %% $Id: TwoLinkPX_desc.tex,v 1.1 2000/12/28 17:23:53 peterg Exp $
% %% $Log: TwoLinkPX_desc.tex,v $
% %% Revision 1.1  2000/12/28 17:23:53  peterg
% %% To RCS
% %%
% %% Revision 1.1  1998/01/19 14:20:07  peterg
% %% Initial revision
% %%
% Revision 1.1  1997/08/15  13:31:00  peterg
% Initial revision
%
% %%%%%%%%%%%%%%%%%%%%%%%%%%%%%%%%%%%%%%%%%%%%%%%%%%%%%%%%%%%%%%%

   The acausal bond graph of system \textbf{TwoLinkPX} is
   displayed in Figure \Ref{TwoLinkPX_abg} and its label
   file is listed in Section \Ref{sec:TwoLinkPX_lbl}.
   The subsystems are listed in Section \Ref{sec:TwoLinkPX_sub}.

This is a heirachical version of the example from Section 10.5 of
"Metamodelling".  It uses the compound components: {\bf ROD}.  {\bf
ROD} is essentially as described in Figure 10.2.
There is no gravity included in this model.

This system has a number of dynamic elements (those corresponding to translation
motion) in derivative causality, thus the system is represnted as a
Differential-Algebraic Equation (Section
\Ref{sec:TwoLinkPX_dae.tex}). Hovever, this is of contrained-state form and
therfore can be written as a set of constrained-state equations (Section
\Ref{sec:TwoLinkPX_cse.tex}). The corresponding ordinary differential
equation is complicated due to the trig functions involved in
inverting the E matrix.

As well as the standard representation the ``robot-form'' equations
appear in Section  \Ref{sec:TwoLinkPX_rfe}. 

%%% Local Variables: 
%%% mode: plain-tex
%%% TeX-master: t
%%% End: 
