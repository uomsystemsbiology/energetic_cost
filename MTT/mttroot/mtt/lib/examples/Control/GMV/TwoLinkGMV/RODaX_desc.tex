% Verbal description for system RODaX (RODaX_desc.tex)
% Generated by MTT on Fri Aug 15 09:53:23 BST 1997.

% %%%%%%%%%%%%%%%%%%%%%%%%%%%%%%%%%%%%%%%%%%%%%%%%%%%%%%%%%%%%%%%
% %% Version control history
% %%%%%%%%%%%%%%%%%%%%%%%%%%%%%%%%%%%%%%%%%%%%%%%%%%%%%%%%%%%%%%%
% %% $Id: RODaX_desc.tex,v 1.1 2000/12/28 17:23:53 peterg Exp $
% %% $Log: RODaX_desc.tex,v $
% %% Revision 1.1  2000/12/28 17:23:53  peterg
% %% To RCS
% %%
% %% Revision 1.1  1998/04/12 15:25:35  peterg
% %% Initial revision
% %%
% Revision 1.1  1997/08/15  09:41:19  peterg
% Initial revision
%
% %%%%%%%%%%%%%%%%%%%%%%%%%%%%%%%%%%%%%%%%%%%%%%%%%%%%%%%%%%%%%%%

   The acausal bond graph of system \textbf{RODaX} is
   displayed in Figure \Ref{RODaX_abg} and its label
   file is listed in Section \Ref{sec:RODaX_lbl}.
   The subsystems are listed in Section \Ref{sec:RODaX_sub}.

{\bf RODaX} is essentially as described in Figure 10.2 of
``Metamodelling''. It has an additional port ``[angle]'' to provide
access to the rod angle.

