% Verbal description for system TwoTanks (TwoTanks_desc.tex)
% Generated by MTT on Mon Jan 12 16:00:15 GMT 1998.

% %%%%%%%%%%%%%%%%%%%%%%%%%%%%%%%%%%%%%%%%%%%%%%%%%%%%%%%%%%%%%%%
% %% Version control history
% %%%%%%%%%%%%%%%%%%%%%%%%%%%%%%%%%%%%%%%%%%%%%%%%%%%%%%%%%%%%%%%
% %% $Id: TwoTanks_desc.tex,v 1.1 2000/12/28 17:39:18 peterg Exp $
% %% $Log: TwoTanks_desc.tex,v $
% %% Revision 1.1  2000/12/28 17:39:18  peterg
% %% To RCS
% %%
% %%%%%%%%%%%%%%%%%%%%%%%%%%%%%%%%%%%%%%%%%%%%%%%%%%%%%%%%%%%%%%%

This is a simple teaching example to illustrate compensator design via
observer/state-feedback methods.

\begin{itemize}
\item Each tank is identical and of unit cross section.
\item Each tank has an identical inflow $f_1 = u$
\item Each outflow is different and given by $\sigma_i v_i$ for $i=1$
  and $i=2$ where $v_i$ is the volume of liquid in tank i and
  $\sigma_i$ the corresponding discharge coefficient. The net outflow
  is thus $y = f_2 = \sigma_1 v_1 + \sigma_2 v_2$.
\item The system states are taken to be:
  \begin{equation}
    x = 
    \begin{pmatrix}
      v_1 \\ v_2
    \end{pmatrix}
  \end{equation}
\end{itemize}




%%% Local Variables: 
%%% mode: latex
%%% TeX-master: t
%%% End: 
