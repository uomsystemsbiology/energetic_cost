% Verbal description for system rc (rc_desc.tex)
% Generated by MTT on Sun Aug 24 11:03:55 BST 1997.

% %%%%%%%%%%%%%%%%%%%%%%%%%%%%%%%%%%%%%%%%%%%%%%%%%%%%%%%%%%%%%%%
% %% Version control history
% %%%%%%%%%%%%%%%%%%%%%%%%%%%%%%%%%%%%%%%%%%%%%%%%%%%%%%%%%%%%%%%
% %% $Id: rc_desc.tex,v 1.1 2000/12/28 17:40:36 peterg Exp $
% %% $Log: rc_desc.tex,v $
% %% Revision 1.1  2000/12/28 17:40:36  peterg
% %% To RCS
% %%
% %% Revision 1.1  1997/08/24 10:27:18  peterg
% %% Initial revision
% %%
% %%%%%%%%%%%%%%%%%%%%%%%%%%%%%%%%%%%%%%%%%%%%%%%%%%%%%%%%%%%%%%%

   The acausal bond graph of system \textbf{rc} is
   displayed in Figure \Ref{rc_abg} and its label
   file is listed in Section \Ref{sec:rc_lbl}.
   The subsystems are listed in Section \Ref{sec:rc_sub}.

The system \textbf{rc} is the simple electrical rc circuit shown in
Figure \Ref{rc_abg}. It can be regarded as a single-input
single-output system with input $e_1$ and output $e_2$.
