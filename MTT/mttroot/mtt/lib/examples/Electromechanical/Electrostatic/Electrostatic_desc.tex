% Verbal description for system Electrostatic (Electrostatic_desc.tex)
% Generated by MTT on Fri Sep 19 18:07:08 BST 1997.

% %%%%%%%%%%%%%%%%%%%%%%%%%%%%%%%%%%%%%%%%%%%%%%%%%%%%%%%%%%%%%%%
% %% Version control history
% %%%%%%%%%%%%%%%%%%%%%%%%%%%%%%%%%%%%%%%%%%%%%%%%%%%%%%%%%%%%%%%
% %% $Id: Electrostatic_desc.tex,v 1.1 1997/11/21 10:59:00 peterg Exp $
% %% $Log: Electrostatic_desc.tex,v $
% %% Revision 1.1  1997/11/21 10:59:00  peterg
% %% Initial revision
% %%
% %%%%%%%%%%%%%%%%%%%%%%%%%%%%%%%%%%%%%%%%%%%%%%%%%%%%%%%%%%%%%%%

   The acausal bond graph of system \textbf{Electrostatic} is
   displayed in Figure \Ref{Electrostatic_abg} and its label
   file is listed in Section \Ref{sec:Electrostatic_lbl}.
   The subsystems are listed in Section \Ref{sec:Electrostatic_sub}.

This is a simple electrostatic speaker using the \textbf{CM}
transducer component together with an electrical \textbf{R} and a
mechanical \textbf{R} and \textbf{C} components to model a compliant
support for the moving plate. See Karnopp, Margolis and Rosenberg
Section 8.2 for a similar example.

