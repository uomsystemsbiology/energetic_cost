% Verbal description for system iTwoLink (iTwoLink_desc.tex)
% Generated by MTT on Mon Nov 17 10:42:48 GMT 1997.

% %%%%%%%%%%%%%%%%%%%%%%%%%%%%%%%%%%%%%%%%%%%%%%%%%%%%%%%%%%%%%%%
% %% Version control history
% %%%%%%%%%%%%%%%%%%%%%%%%%%%%%%%%%%%%%%%%%%%%%%%%%%%%%%%%%%%%%%%
% %% $Id: iTwoLink_desc.tex,v 1.1 2000/12/28 17:56:50 peterg Exp $
% %% $Log: iTwoLink_desc.tex,v $
% %% Revision 1.1  2000/12/28 17:56:50  peterg
% %% To RCS
% %%
% %% Revision 1.3  1998/01/19 10:08:21  peterg
% %% Added comment about linearisation point
% %%
% %% Revision 1.2  1998/01/19 09:57:26  peterg
% %% Added a discussion of the relevance of G(s).
% %%
% Revision 1.1  1997/12/09  16:53:27  peterg
% Initial revision
%
% %%%%%%%%%%%%%%%%%%%%%%%%%%%%%%%%%%%%%%%%%%%%%%%%%%%%%%%%%%%%%%%

   The acausal bond graph of system \textbf{iTwoLink} is
   displayed in Figure \Ref{iTwoLink_abg} and its label
   file is listed in Section \Ref{sec:iTwoLink_lbl}.
   The subsystems are listed in Section \Ref{sec:iTwoLink_sub}.

This example illustrates the inversion of  two link manipulator
dynamics using two identical  simple mass-spring-damper systems as
specification systems.

The velocities $\omega_1=\omega_2$ specified by the specification
systems are given in Figure \Ref{fig:iTwoLink_odeso.ps-iTwoLink-t1s}
together with the input defined in Section \Ref{sec:iTwoLink_input.txt}.
The torques $\tau_1$ and $\tau_2$ required to give the these
velocities specified by the specification system are given in Figures
\Ref{fig:iTwoLink_odeso.ps-iTwoLink-t1} and
\Ref{fig:iTwoLink_odeso.ps-iTwoLink-t2} respectively.

The corresponding velocity/torque diagrams for joints 1 and 2 appear in
Figures \Ref{fig:iTwoLink_odeso.ps-iTwoLink-t1s:iTwoLink-t1}
\Ref{fig:iTwoLink_odeso.ps-iTwoLink-t2s:iTwoLink-t2} respectively.
Such diagrams can be used for actuator sizing in terms of torque,
velocity and power.

 
This non-linear system can be linearised (about the various
configurations) and small-signal frequency response methods applied.
For example, the four transfer functions $G_11$ to $G_22$ in Section
\Ref{sec:iTwoLink_tf} (representing the system linearised about zero
angles and velocities), give the small-signal relations between the
two spec. torques and the required system torques. Used together with
$G_31$ and $G_42$ (relating the spec. torques and the joint
velocities) gives, in principle, a method for evaluating actuator
requirements (for small signals) as a function of frequency.

