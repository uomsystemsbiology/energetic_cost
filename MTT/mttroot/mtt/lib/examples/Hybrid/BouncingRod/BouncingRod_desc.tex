% Verbal description for system BouncingRod (BouncingRod_desc.tex)
% Generated by MTT on Sun Jul 6 17:21:23 BST 1997.

% %%%%%%%%%%%%%%%%%%%%%%%%%%%%%%%%%%%%%%%%%%%%%%%%%%%%%%%%%%%%%%%
% %% Version control history
% %%%%%%%%%%%%%%%%%%%%%%%%%%%%%%%%%%%%%%%%%%%%%%%%%%%%%%%%%%%%%%%
% %% $Id: BouncingRod_desc.tex,v 1.3 2002/09/29 15:31:14 geraint Exp $
% %% $Log: BouncingRod_desc.tex,v $
% %% Revision 1.3  2002/09/29 15:31:14  geraint
% %% Updated cross-reference.
% %%
% %% Revision 1.2  1999/12/20 09:36:24  peterg
% %% *** empty log message ***
% %%
% Revision 1.1  1997/07/06  16:35:20  peterg
% Initial revision
%
% %%%%%%%%%%%%%%%%%%%%%%%%%%%%%%%%%%%%%%%%%%%%%%%%%%%%%%%%%%%%%%%

The acausal bond graph of system \textbf{BouncingRod} is displayed in
Figure \Ref{BouncingRod_abg} and its label file is listed in Section
\Ref{sec:BouncingRod_lbl}. 
 The subsystems are listed in Section
\Ref{sec:BouncingRod_sub}.

The system consists of a uniform rod of mass 1kg, length 2m (and
therefore of inertia about the mass centre of $\frac{1}{3}
\text{kgm}^2$. The rod is released at an angle of $\frac{\pi}{4}$ from
the vertical, the mass centre is $10\text{m}$ above the ground and all
velocities are initially zero. The gravitational constant is taken as unity.

The ground is modeled as an ideal compliance in the vertical
direction with compliance of $0.1 \text{mN}^{-1}$ and it is assumed
that contact takes place at the rod tips only. There is no
horizontal resistance to motion. This idealised setup is modeled by a
two {\bf CSW} components, one for each rod tip, modulated by the
height of each rod tip above the ground: each  {\bf CSW} is off when
the corresponding height is positive.

The system was simulated for 100 time units and the resultant height
of each tip is plotted in Figure \ref{fig:BouncingRod_odeso-noargs.ps}. The rod was
released at zero velocity from a height of ten units and at an angle
of $\frac{\pi}{4}$ radians to the vertical. The oscillations in height
are due to rod rotation about its mass centre.  The bounce height
changes due to energy transfer to and from the rod rotation about its
mass centre.




%%% Local Variables: 
%%% mode: plain-tex
%%% TeX-master: "yes"
%%% End: 
