% Verbal description for system Weirs (Weirs_desc.tex)
% Generated by MTT on Thu Jul 3 10:27:26 BST 1997.

% %%%%%%%%%%%%%%%%%%%%%%%%%%%%%%%%%%%%%%%%%%%%%%%%%%%%%%%%%%%%%%%
% %% Version control history
% %%%%%%%%%%%%%%%%%%%%%%%%%%%%%%%%%%%%%%%%%%%%%%%%%%%%%%%%%%%%%%%
% %% $Id: Weirs_desc.tex,v 1.2 2002/09/29 14:19:44 geraint Exp $
% %% $Log: Weirs_desc.tex,v $
% %% Revision 1.2  2002/09/29 14:19:44  geraint
% %% Updated cross-reference.
% %%
% %% Revision 1.1  2000/12/28 17:47:43  peterg
% %% To RCS
% %%
% Revision 1.1  1997/09/11  10:31:48  peterg
% Initial revision
%
% %%%%%%%%%%%%%%%%%%%%%%%%%%%%%%%%%%%%%%%%%%%%%%%%%%%%%%%%%%%%%%%

   The acausal bond graph of system \textbf{Weirs} is
   displayed in Figure \Ref{Weirs_abg} and its label
   file is listed in Section \Ref{sec:Weirs_lbl}.
   The subsystems are listed in Section \Ref{sec:Weirs_sub}.

%Each weir is modeled by two {\bf ISW} components: one for flow
%left-right and one for flow right-left. The switching is such that
%they become non-return valves when the left hand (respectively
%right-hand) level reaches an appropriate value. The parameters appear
%in Section \Ref{sec:Weirs_numpar.txt} and the switching conditions in
%Section \Ref{sec:Weirs_input.txt}.

Aircraft fuel tanks are often fitted with baffles to reduce fuel
slosh. A simple model relating to such a system is shown in Figure
\Ref{Weirs_abg} which corresponds to a single tank containing two
dividing weirs. Liquid with flow rate $f$ enters the left-hand
compartment; liquid leaks out of the centre compartment at a flow rate
determined by gravity and the properties of the corresponding orifice.

The Bond Graph appearing in Figure \Ref{Weirs_abg} represents each of
the three compartments by a \textbf{C} component (labelled tank1 to
tank3), the corresponding pressures are measured by the \textbf{SS}
elements p1--p3. The leak is represented by the \textbf{R} component
labelled leak. The flows over the two weirs are represented by the four
\textbf{ISW} components; each weir has a separate \textbf{ISW}
component for each flow direction. Each \textbf{ISW} component is
switched by the appropriate level.

%It could be argued that, when switched on, each \textbf{ISW} component
%corresponds to flow inertia; but it is admitted that ideal
%\textbf{Sw} components would give a simpler approximation in this
%case. So the modeller has the choice of having a simple simulation
%problem but with four extra states, or a complex simulation with model
%switching but without the four extra states.

The system was simulated for 20 time units and the resultant level of
each tank partition is plotted in Figure \Ref{fig:Weirs_odeso-noargs.ps}. Each
partition has unit cross section, and the two weir heights are $1$ and
$2$ respectively; the inflow $f$ is given by:
\begin{equation}
  f = 
  \begin{cases}
    1 & \text{if $t \le 10$}\\
    0 & \text{if $t > 10$}
  \end{cases}
\end{equation}
and the leak resistance is linear with flow resistance 5.


%%% Local Variables: 
%%% mode: plain-tex
%%% TeX-master: t
%%% End: 
