% Verbal description for system HeatedRod (HeatedRod_desc.tex)
% Generated by MTT on Thu Sep 4 18:05:09 BST 1997.

% %%%%%%%%%%%%%%%%%%%%%%%%%%%%%%%%%%%%%%%%%%%%%%%%%%%%%%%%%%%%%%%
% %% Version control history
% %%%%%%%%%%%%%%%%%%%%%%%%%%%%%%%%%%%%%%%%%%%%%%%%%%%%%%%%%%%%%%%
% %% $Id: HeatedRod_desc.tex,v 1.1 2000/12/28 18:12:41 peterg Exp $
% %% $Log: HeatedRod_desc.tex,v $
% %% Revision 1.1  2000/12/28 18:12:41  peterg
% %% To RCS
% %%
% %% Revision 1.1  1997/09/11 16:16:50  peterg
% %% Initial revision
% %%
% %%%%%%%%%%%%%%%%%%%%%%%%%%%%%%%%%%%%%%%%%%%%%%%%%%%%%%%%%%%%%%%

\begin{table}[htbp]
  \begin{center}
    \leavevmode
    \begin{tabular}{l l l}
      \hline
      Parameter & Symbol & Value  \\
      \hline
      Length & $L_r$ & 1m \\
      Diameter & $D_r$ & 1mm \\
      Resistivity & $\rho$ & $1.68\times10^{-9} \Omega$m \\
      Thermal conductivity & $\sigma$ & 390 W$\text{m}^{-1}$\\
      Thermal capacity & $\kappa$ & 380 J$\text{m}^{-3}$\\
      \hline
    \end{tabular}
    \caption{Heated rod parameters}
    \label{tab:rod}
  \end{center}
\end{table}

 System \textbf{HeatedRod} is a model of a well-insulated rod of copper with an
 electric current passing through it which warms it up. The two ends of
 the rod are fixed at ambient temperature; this is where all the heat
 loss occurs. 
 
 This example introduces the idea of the {\bf FP}, \textbf{RT} and
 \textbf{CT} components in the context of thermal conduction.
 
 The model is similar to that described in chapter 8 of \citeN{Cel91}.
 However, instead of representing the thermal resistance by {\bf RS}
 components and reinserting the entropy flow, the {\bf RT} component
 uses two {\bf FP} components to convert from true to pseudo bonds and
 back again. Similary, the thermal capacity is modelled by the {\bf
   CT} component.
 
 This distributed system (which strictly speaking has a partial
 differential equation model) is approximated by an ordinary
 diffferential equation model by modelling the system by a number of
 discrete segments of length $\Delta x$. Each segment model consists
 of two conceptual parts.
 \begin{itemize}
 \item An ideal lump of copper with no thermal resistance but with the
   normal attributes of electrical resistance (modelled by the
   \textbf{RS} component and thermal capacity (modelled by the
   \textbf{CF} component).
 \item A thin lump wtih thermal resistance but no thermal capacity or
   electrical resistance (modeled by the \textbf{RT} component).
 \end{itemize}
At this level of the hierarchy, all bonds are true energy bonds and
thus energy conservation is assured. Note that the \textbf{RS}
component correctely transforms electrical to thermal energy.

The system was simulated with a total of nine lumps whilst passing a
current of 1A though the rod for a total of 10s. The initial
temperature and the end temperatures were all set at 300K.
