% -*-latex-*- Put EMACS into LaTeX-mode
% Verbal description for system CantileverBeam (CantileverBeam_desc.tex)
% Generated by MTT on Mon Apr 19 07:04:54 BST 1999.

% %%%%%%%%%%%%%%%%%%%%%%%%%%%%%%%%%%%%%%%%%%%%%%%%%%%%%%%%%%%%%%%
% %% Version control history
% %%%%%%%%%%%%%%%%%%%%%%%%%%%%%%%%%%%%%%%%%%%%%%%%%%%%%%%%%%%%%%%
% %% $Id: CantileverBeam_desc.tex,v 1.1 2000/12/28 17:58:27 peterg Exp $
% %% $Log: CantileverBeam_desc.tex,v $
% %% Revision 1.1  2000/12/28 17:58:27  peterg
% %% To RCS
% %%
% %% Revision 1.1  1999/05/18 04:01:50  peterg
% %% Initial revision
% %%
% %%%%%%%%%%%%%%%%%%%%%%%%%%%%%%%%%%%%%%%%%%%%%%%%%%%%%%%%%%%%%%%

The acausal bond graph of system \textbf{CantileverBeam} is displayed in
Figure \Ref{fig:CantileverBeam_abg.ps} and its label file is listed in
Section \Ref{sec:CantileverBeam_lbl}.  The subsystems are listed in Section
\Ref{sec:CantileverBeam_sub}.
   
This example represents the dynamics of a uniform beam with one fixed
and one free end.  The beam is approximated by 20 equal lumps using
the Bernoulli-Euler approximation with damping. 
The input is the angular velocity of the fixed end, the output is the
linear velocity of the free end.

The system parameters are given in Section
\Ref{sec:CantileverBeam_numpar.tex}. Note that the numer of ban
segments has been set to 21. 

 The system has 20 states (10
modes of vibration), 1 inputs and 1 outputs.

The first 5 vibration frequencies are given in Table \ref{tab:freq}
togtherr with the theoretical (based on the Bernoulli-Euler beam with
the same values of $EI$ and $\rho A$. 
\begin{table}[htbp]
  \begin{center}
    \begin{tabular}{||l|l|l||}
      \hline
      \hline
      Mode & Frequency & Theoretical frequency\\
      \hline
      1 &  76.14 &  76.14\\
      2 & 477.11 & 484.50\\
      3 &1330.62 &1334.55\\
      4 &2586.77 &2617.19\\
      5 &4225.14 &4323.77\\
       \hline
      \hline
    \end{tabular}
    \caption{Mode frequencies (rad $s^{-1}$)}
    \label{tab:freq}
  \end{center}
\end{table}





