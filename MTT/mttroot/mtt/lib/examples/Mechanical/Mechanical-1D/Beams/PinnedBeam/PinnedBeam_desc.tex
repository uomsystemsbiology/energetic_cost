% -*-latex-*- Put EMACS into LaTeX-mode
% Verbal description for system PinnedBeam (PinnedBeam_desc.tex)
% Generated by MTT on Mon Apr 19 07:04:54 BST 1999.

% %%%%%%%%%%%%%%%%%%%%%%%%%%%%%%%%%%%%%%%%%%%%%%%%%%%%%%%%%%%%%%%
% %% Version control history
% %%%%%%%%%%%%%%%%%%%%%%%%%%%%%%%%%%%%%%%%%%%%%%%%%%%%%%%%%%%%%%%
% %% $Id: PinnedBeam_desc.tex,v 1.3 2000/12/28 17:59:05 peterg Exp $
% %% $Log: PinnedBeam_desc.tex,v $
% %% Revision 1.3  2000/12/28 17:59:05  peterg
% %% To RCS
% %%
% %% Revision 1.2  1999/11/24 22:17:26  peterg
% %% Updated to correspond to Reza's beam
% %%
% %% Revision 1.1  1999/10/11 05:08:14  peterg
% %% Initial revision
% %%
% %% Revision 1.1  1999/05/18 04:01:50  peterg
% %% Initial revision
% %%
% %%%%%%%%%%%%%%%%%%%%%%%%%%%%%%%%%%%%%%%%%%%%%%%%%%%%%%%%%%%%%%%

The acausal bond graph of system \textbf{PinnedBeam} is displayed in
Figure \Ref{fig:PinnedBeam_abg.ps} and its label file is listed in
Section \Ref{sec:PinnedBeam_lbl}.  The subsystems are listed in Section
\Ref{sec:PinnedBeam_sub}.
   
This example represents the dynamics of a uniform beam with two pinned
ends. The left-hand end is driven by a torque input and the
corresponding collocated angular velocity is measured.  The beam is
approximated by 20 equal lumps using the Bernoulli-Euler.  Because the
two end lumps have different causality to the rest of the beam lumps,
they are represented seperately.  The system has 40 states (20 modes
of vibration), 1 input and 1 output.

\begin{table}[htbp]
  \begin{center}
    \begin{tabular}{|l|l|}
      \hline
      Name & Value\\
      \hline
      Beam Length, $L$       & 0.60 m\\
      Beam Width $w$             & 0.05 m\\
      Beam Thickness $t_b$   & 0.003\\
      Young's Modulus  $E$       & $68.94 \times 10^9$ \\
      Density          $\rho$    & 2712.8\\
      \hline
      Derived quantities & \\
      \hline
      $EI$                     & 7.76\\
      $\rho A$   & 0.40692 \\
      \hline
    \end{tabular}
    \caption{Beam parameters}
    \label{tab:beam}
  \end{center}
\end{table}


The beam was made of aluminium with physical dimensions and constants
given in Table \ref{tab:beam}. The derived beam constants are given by the
formulae:
\begin{equation}
  \label{eq:formulae}
  \begin{align}
    EI &= E \times w \frac{1}{12} t_b^3\\
    \rho A &= \rho \times w t_b
  \end{align}
\end{equation}

The system parameters are also given in Section
\Ref{sec:PinnedBeam_numpar.tex}.


\begin{table}[htbp]
  \begin{center}
    \begin{tabular}{||l|l|l|l|l||}
      \hline
      \hline
Index   & $f_r$ (theory) & $f_r$ (model)& $f_a$ (theory) & $f_a$ (model) \\ 
\hline
1       & 19.05         & 19.01         & 29.72         & 31.28\\ 
2       & 76.24         & 75.57         & 96.50         & 100.80\\ 
3       & 171.58        & 168.29        & 200.73        & 208.20\\ 
4       & 304.76        & 294.89        & 344.13        & 350.88\\ 
5       & 476.34        & 452.25        & 524.98        & 525.23\\ 
      \hline
      \hline
    \end{tabular}
    \caption{Resonant and anti-resonant frequencies (Hz)}
    \label{tab:freq}
  \end{center}
\end{table}

Standard modal analysis give the theoretical system resonant
frequencies $f_r$ (based on the Bernoulli-Euler beam with the same values of
$EI$ and $\rho A$). The system anti-resonances $f_a$ correspond to those of
the \emph{inverse} system with reversed causality, that the driven
pinned end is replaced by a clamped end; again modal analysis of the
inverse system gives the system anti resonances. The model and
theoretical values are compared in Table \ref{tab:freq} for the first
5 modes. (This table was generated using the script MakeFreqTable.m)

