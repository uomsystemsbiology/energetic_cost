% Verbal description for system TwoLinkxyn (TwoLinkxyn_desc.tex)
% Generated by MTT on Fri Jun 13 16:30:23 BST 1997.

% %%%%%%%%%%%%%%%%%%%%%%%%%%%%%%%%%%%%%%%%%%%%%%%%%%%%%%%%%%%%%%%
% %% Version control history
% %%%%%%%%%%%%%%%%%%%%%%%%%%%%%%%%%%%%%%%%%%%%%%%%%%%%%%%%%%%%%%%
% %% $Id: TwoLinkxyn_desc.tex,v 1.2 2000/12/28 18:03:12 peterg Exp $
% %% $Log: TwoLinkxyn_desc.tex,v $
% %% Revision 1.2  2000/12/28 18:03:12  peterg
% %% To RCS
% %%
% Revision 1.1  1998/01/06  17:36:33  peterg
% Initial revision
% %%%%%%%%%%%%%%%%%%%%%%%%%%%%%%%%%%%%%%%%%%%%%%%%%%%%%%%%%%%%%%%

   The acausal bond graph of system \textbf{TwoLinkxyn} is
   displayed in Figure \Ref{TwoLinkxyn_abg} and its label
   file is listed in Section \Ref{sec:TwoLinkxyn_lbl}.
   The subsystems are listed in Section \Ref{sec:TwoLinkxyn_sub}.

This system is identical to  \textbf{twolink} except that there are
now two non-collocated input-output pairs: The torque input to joint 1
-- x velocity of the tip and the torque input to joint 2
-- y velocity of the tip. 

It uses two compound components: {\bf ROD} and {\bf GRAV}.  {\bf ROD}
is essentially as described in Figure 10.2 of "Metamodelling" and {\bf
GRAV} represents gravity by a vertical acceleration as in Section
10.9 of "Metamodelling"

%%% Local Variables: 
%%% mode: plain-tex
%%% TeX-master: t
%%% End: 
